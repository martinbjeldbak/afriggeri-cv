%!TEX TS-program = xelatex
%\documentclass[print]{friggeri-cv}
\documentclass[]{friggeri-cv}

\begin{document}
\header{elias}{obeid}
       {datalogistuderende}


% In the aside, each new line forces a line break
\begin{aside}
  \section{om}
    Elias Khazen Obeid
    7. februar 1991
    Sigrid Undsets Vej 
    9220 Aalborg
    Danmark
    ~
    \textbf{mobil} 71 20 29 91
    \href{mailto:e.k.obeid@gmail.com}{e.k.obeid@gmail.com}
    \href{http://eliasobeid.dk}{eliasobeid.dk}
    \href{http://facebook.com/eliaskhazenobeid}{fb://eliasobeid}
    \href{http://www.linkedin.com/in/eliasobeid}{in://eliasobeid}
  \section{sprog}
    tresproget dansk/english/arabisk
  \section{programmering og markup}
    JavaScript, Java,
    Python, C, C\#,
    Ruby og Ruby on Rails,
    HTML, CSS og \LaTeX{}
\end{aside}


\section{uddannelse}

\begin{entrylist}
  \entry
    {2011–2014}
    {B.Sc. {\normalfont i Datalogi}}
    {Aalborg Universitet}
    {Seks semestre af 4 måneders varighed \\
     projektarbejde (15 ETCS) og tre kurser (af 5 ETCS) pr. semester}
  \entry
    {2011–2011}
    {Brandmandsuddannelse}
    {Beredskabsstyrelsen Nordjylland}
    {Brand, Redning, CBRNE og Kommunikation}
  \entry
    {2007–2010}
    {Student}
    {Aalborg Tekniske Gymnasium}
    {Kommunikation og Medie}
  \entry
    {1998–2007}
    {Folkeskoleelev}
    {Mellervangskolen}
    {0. til 9. klasse}
\end{entrylist}

\section{erhvervserfaring}

\begin{entrylist}
  \entry
    {siden 2014}
    {Studenterudvikler}
    {Falck Healthcare}
    {Udvikling af system (DentalCare), som Falck Healthcare tilbyder sine kunder \\
     Hovedsagligt webudvikling i Ruby on Rails}
  \entry
    {08–12 2013}
    {Hjælpelærer}
    {Aalborg Universitet}
    {Hjælpelærer i Imperativ Programmering med C som programmeringssprog}
  \entry
    {2012–2014}
    {Ambassadør}
    {Aalborg Universitet}
    {Ambassadør for Datalogi}
  \entry
    {08–11 2011}
    {Telefonsælger}
    {LN Eurocom}
    {Salg af avis for Nordjyske Stiftstidende}
  \entry
    {02–07 2011}
    {Menig (værnepligtig)}
    {Beredskabsstyrelsen Nordjylland}
    {Brandmandsuddannelse og beredskabsvagt}
  \entry
    {2010–2011}
    {Butiksmedarbejder}
    {Twenty4-7, Aalborg}
    {Kundebetjening, rengøring og vareopfyldning}
  \entry
    {2008–2010}
    {Butiksmedarbejder/Chauffør}
    {La Pronto Pizza, Aalborg}
    {Kundebetjening, rengøring, madlavning og udbringning}
\end{entrylist}

\section{interesser og fritid}

Jeg har altid interesseret mig for teknologi. Jeg begyndte at læse datalogi for at lære mere om programmering og generelt IT. Jeg er blevet specielt fanget af emner som maskinlæring. Jeg vil fokusere på maskinlæring i min kandidat, og jeg har fanget mig selv i, at synes at kryptografi er yderst spændende. 

Jeg har stadig lang vej endnu, men programmering har altid virket meget naturligt og intuitivt for mig. Jeg tog et kort kursus med programmering på c-niveau i gymnaiset, og det blev jeg fanget af.

Jeg holder af at dyrke motion og sport. Jeg elsker at slappe af med en god film i godt selskab. Hvis jeg begynder på en god bog, så har jeg svært ved at slippe den, inden jeg er nået til sidste side.

\newpage

\section{kompetencer}
Jeg er meget lærenem og nysgerrig. Jeg kan nemt og hurtigt sætte mig ind i nye ting. Jeg har stor interesse for programmering, som kommer meget naturligt til mig.
Jeg har erfaring med CMS-systemer som WordPress og Joomla. Versionkontrolsystemer arbejder jeg med til hverdag. Her er det især Git jeg arbejder med, men jeg har også arbejdet lidt med SVN.
Derudover har jeg kendskab til operativsystemer, især Windows og Ubuntu.

\section{anden aktivitet}

\begin{entrylist}
  \entry
  {2013–2014}
  {Medlem af Studienævn for Datalogi}
  {Aalborg Universitet}
  {Diskussion af meritansøgninger, studieordninger, dispensationsansøgninger, kvalitetssikring af uddannelsen og lignende.}
  \entry
  {siden 2013}
  {Opsætning af hjemmeside med WordPress}
  {Beredskabscenter Aalborg}
  {Frivilligt arbejde. Opsat hjemmeside til frivillige tilknyttet Beredskabscenter Aalborg.
  Vedligeholder fortsat \href{http://frv-aalborg.dk}{frv-aalborg.dk}.}
  \entry
  {09–10 2013}
  {Rusintruktør}
  {Aalborg Universitet}
  {Rusintruktør for nye studerende}
  \entry
  {2012–2014}
  {Frivillig Mentor }
  {Ungdommens Røde Kors}
  {Del af projekt \href{http://www.urk.dk/solskinsunge/}{Solskinsunge}, hvor vi holdte sociale arrangementer med unge fra Tornhøjskolen i Aalborg øst}
  \entry
  {09–10 2012}
  {Rusintruktør}
  {Aalborg Universitet}
  {Rusintruktør for nye studerende}
  \entry
  {siden 2011}
  {Frivillig brandmand}
  {Beredskabscenter Aalborg}
  {Vedligeholdelse af brandmandsuddannelsen}
  \entry
  {siden 2009}
  {B Kørekort}
  {}{}
\end{entrylist}

\end{document}
